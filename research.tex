%-------------------------------------------------------------------------------
%	SECTION TITLE
%-------------------------------------------------------------------------------
\cvsection{Publications}


%-------------------------------------------------------------------------------
%	CONTENT
%-------------------------------------------------------------------------------
\begin{cventries}

%---------------------------------------------------------
  \paperentry
    {SGX-Shield: Enabling Address Space Layout Randomization for SGX Programs} % Role
    {\textbf{Jaebaek Seo}, Byoungyoung Lee, Sungmin Kim, Ming-Wei Shih, Insik Shin, Dongsu Han, and Taesoo Kim, } % Event
    { To be appeared in \textbf{proceedings of NDSS 2017}, Acceptance ratio: 68/423=16.1\% } % Date(s)
    {
        \begin{cvitems} % Description(s)
        \item {SGX-Shield is a system (compiler toolchains and runtime support)
            to enable ASLR (Address Space Layout Randomization) for SGX programs.}
        \item {Jaebaek Seo alone implemented all compiler toolchains including
                LLVM backends, static linker and dynamic loader/linker and runtime
            support including libraries and memory layout.}
        \item {\url{https://github.com/jaebaek/SGX-Shield}}
        \end{cvitems}
    }

  \paperentry
    { FLEXDROID: Enforcing In-App Privilege Separation in Android } % Role
    { \textbf{Jaebaek Seo}, Daehyeok Kim, Donghyun Cho, Taesoo Kim, Insik Shin, }
    { \textbf{Proceedings of NDSS 2016}, Acceptance ratio: 60/389=15.4\% } % Date(s)
    {
        \begin{cvitems} % Description(s)
        \item {FlexDroid is an extension of Android permission system
            to support in-app privilege separation.}
        \item {Jaebaek Seo alone engineered memory permission part in kernel,
            Dalvik JVM, Android framework, dynamic loader/linker.}
        \item {\url{https://github.com/flexdroid}}
        \end{cvitems}
    }

  \paperentry
    {Optimal Real-Time Scheduling on Two-Type Heterogeneous Multicore Platforms} % Role
    {Hoon Sung Chwa, Jaebaek Seo, Jinkyu Lee, Insik Shin,} % Date(s)
    {Proceedings of the 36th IEEE Real-Time Systems Symposium (RTSS '15)} % Date(s)
    {
        \begin{cvitems} % Description(s)
        \item {Jaebaek Seo contributed to prove mathmatical theorems.}
        \item{This work is published in Proceedings of the 36th IEEE Real-Time
                Systems Symposium (RTSS 2015, Acceptance ratio: 34/151=22.5\%)}
            \item {Jaebaek Seo is the second author of the paper.}
            \end{cvitems}
    }

%---------------------------------------------------------
\end{cventries}
